%%%%%%%%%%%%%%% COPYRIGHT ANTOINE HUGOUNET & ETHEL VILLENEUVE
%%%%%%%%%%%%%%%%%%%%%%%%%%%%%%%%%%%%%%%%%%%%%%%%%%%%%%%
%%%%%%%%%%%%%%%%%%%%%%%%%%%%%%%%%%%%%%%%%%%%%%%%%%%%%%%


\documentclass[a4paper, twoside, 11pt]{report}

\usepackage[utf8]{inputenc}
\usepackage[T1]{fontenc}
\usepackage[english]{babel}
\usepackage[top= 120pt, left=80pt, right=80pt]{geometry} %marges
\usepackage{setspace} %interlignage
\usepackage{url}
\usepackage{graphicx}
\usepackage{lmodern} 
\usepackage{array}
\usepackage{csquotes}
\usepackage[numbers,square]{natbib}
\usepackage{soul}
\usepackage{hyperref}
\usepackage{amsthm}
\usepackage{color}
\usepackage[usenames,dvipsnames,svgnames,table]{xcolor}
\usepackage{adjustbox}
\usepackage{amssymb}
\usepackage{amsmath}
\usepackage{dsfont}
%%\usepackage{braket}
\usepackage{physics}
\usepackage{amsfonts}
\usepackage[numbers,square]{natbib}
\usepackage{multirow}
\usepackage{listings}

%%%%%% COULEURS CODE C++
\lstdefinestyle{customc}{
  belowcaptionskip=1\baselineskip,
  breaklines=true,
  frame=L,
  xleftmargin=\parindent,
  language=C,
  showstringspaces=false,
  basicstyle=\footnotesize\ttfamily,
  keywordstyle=\bfseries\color{red},
  commentstyle=\itshape\color{gray},
  identifierstyle=\color{NavyBlue},
  stringstyle=\color{black},
}

\lstdefinestyle{customasm}{
  belowcaptionskip=1\baselineskip,
  frame=L,
  xleftmargin=\parindent,
  language=[x86masm]Assembler,
  basicstyle=\footnotesize\ttfamily,
  commentstyle=\itshape\color{purple!40!black},
}

\lstset{escapechar=@,style=customc}

%%%%%% STYLES DE THÉORÈMES

\newtheoremstyle{theorem}%	Name
  {}%	Space above
  {}%	Space below
  {}%	Body font
  {}%	Indent amount
  {\bfseries}%	Theorem head font
  {.}%	Punctuation after theorem head
  { }%	Space after theorem head, ' ', or \newline
  {}%	Theorem head spec (can be left empty, meaning `normal')

\newtheoremstyle{exemple}%	Name
  {}%	Space above
  {}%	Space below
  {\color{Gray}\itshape}%	Body font
  {}%	Indent amount
  {\color{Gray}\itshape}%	Theorem head font
  {.}%	Punctuation after theorem head
  { }%	Space after theorem head, ' ', or \newline
  {}%	Theorem head spec (can be left empty, meaning `normal')

\newtheoremstyle{remark}%	Name
  {}%	Space above
  {}%	Space below
  {\itshape}%	Body font
  {}%	Indent amount
  {\bfseries}%	Theorem head font
  {.}%	Punctuation after theorem head
  { }%	Space after theorem head, ' ', or \newline
  {}%	Theorem head spec (can be left empty, meaning `normal')
  
%%%%%% DÉCLARATION DES THÉORÈMES

\theoremstyle{theorem}
\newtheorem{theorem}{Theorem}[section]
\newtheorem{lemme}{Lemma}[section]
\newtheorem{proposition}{Proposition}[section]
\newtheorem{definition}{Definition}[section]

\theoremstyle{remark}
\newtheorem{remark}{Remark}[chapter]

\theoremstyle{exemple}
\newtheorem*{exemple}{Example}


%%%%%% COMMANDES QUI SIMPLIFIENT LA VIE

\newcommand{\legende}[1]{
\begin{center}
	\begin{minipage}{12cm}
		\begin{center}
			\textit{\textcolor{WildStrawberry!30}{#1}}
		\end{center}
	\end{minipage}
\end{center}}

\newcommand{\sherlock}[2]{
	\begin{equation}
		\textcolor{WildStrawberry}{#1}
	\end{equation}
	\legende{#2}
}
	
\newcommand{\sherlocked}[1]{
	\begin{equation}
		\textcolor{WildStrawberry}{#1}
	\end{equation}
}

	
\newcommand{\defSherlock}[3]{
	\begin{definition}[\textbf{#1}]
		\sherlock{#2}{#3}
	\end{definition}
}

\newcommand{\defSherlocked}[2]{
	\begin{definition}[\textbf{#1}]
		\sherlocked{#2}
	\end{definition}
}	

\newcommand{\propSherlock}[3]{
	\begin{proposition}[\textbf{#1}]
		\sherlock{#2}{#3}
	\end{proposition}
}

\newcommand{\propSherlocked}[2]{
	\begin{proposition}[\textbf{#1}]
		\sherlocked{#2}
	\end{proposition}
}

\newcommand{\textSherlocked}[1]{
	\begin{center}
		\textcolor{WildStrawberry}{#1}
	\end{center}
}

\newcommand{\N}{\mathbb{N}}
\newcommand{\Z}{\mathbb{Z}}
\newcommand{\R}{\mathbb{R}}
\newcommand{\C}{\mathbb{C}}


%%%%%%%%%%%%%%%%%%%%%%%%%%%%%%%%%%%%%%%%%%%%%%%%%%%%%%%
%%%%%%%%%%%%%%%%%%%%%%%%%%%%%%%%%%%%%%%%%%%%%%%%%%%%%%%
%%%%%%%%%%%%%%%%%%%%%%%%%%%%%%%%%%%%%%%%%%%%%%%%%%%%%%%

\title{FYS3150\\Project 3 - }
\author{Hugounet, Antoine \& Villeneuve, Ethel}
\date{September 2017 \\University of Oslo \\ \url{https://github.com/kryzar/Perseids.git}}



\begin{document}
\selectlanguage{english}
\maketitle
	
	
\begin{abstract}

	\paragraph{}
	
\end{abstract}


\tableofcontents


\chapter*{Introduction}
\addcontentsline{toc}{chapter}{Introduction}

    \paragraph{}
    

\chapter{Theory}
    
    \paragraph{}In a first place, we will begin with an Earth-Sun system with the Earth orbiting around the Sun to test a simple algorithm and then we will add the other planets to have a simulation of the complete Solar System. 

    \section{Earth-Sun system}
        \subsection{Physical conditions of the system}
            \paragraph{}The only force applied to this system is the gravity. According to the Newton's law, we have 
                \begin{equation*}
                F_G = \frac{GM_{\odot}M_{\oplus}}{r^2}
                \end{equation*}
            with $F_G$ the gravitational force, $G$ the gravitational constant ($G=6.674 \times 10^{-11} N.m^2.kg^{-2}$), $M_{\odot}$ the mass of the Sun, $M_{\oplus}$ the mass of the Earth and $r$ the distance between the Earth and the Sun.\\
            We will neglect the motion of the Sun here as the mass of the Sun is much larger than the mass of the Earth (more than $300 \times M_{\oplus}$). We want to establish the motion of the Earth around the Sun. Moreover, we will assume that the orbit of the Earth around the Sun is coplanar in the $xy$-plane.
        \subsection{}
            \paragraph{}The Newton's second law of motion is given by $F=M \times a$ with $F$ the force applied to the system, $M$ the mass of the body concerned and $a$ the acceleration. Applied to our case, we have $F_G = M_{\oplus} \times a$ which can be written in two equations : 
                \begin{align*}
                    F_{G,x} &= M_{\oplus} \frac{d^2 x}{dt^2} \\
                    F_{G,y} &= M_{\oplus} \frac{d^2 y}{dt^2}
                \end{align*}
            or
                \begin{align}
                    \frac{d^2 x}{dt^2} &= \frac{F_{G,x}}{M_{\oplus}} \\
                    \frac{d^2 y}{dt^2} &= \frac{F_{G,y}}{M_{\oplus}}.
                \end{align}
            with $F_{G,x}$ and $F_{G,y}$ the components of the gravitational force.
            \paragraph{} We will not use the SI units but the Astronomical units (AU) for the distance (with $1$AU$=$average distance Earth-Sun $=1.5 \times 10^11$m) and years for time units. 
            
    
    \section{Three-body problem}


    \section{The complete Solar system}

    

\chapter{Implementation}

    \section{Earth-Sun system}
        \paragraph{}
    
    \section{Adding one planet}
    
    
    \section{The complete Solar system}
    
    

\chapter{Results}

    \section{}
    
    \section{}
    
    \section{}
    
    

\chapter*{Conclusion}
\addcontentsline{toc}{chapter}{Conclusion}

    \paragraph{}
    
    
    
    
    
    
    
    
    
    
    
    
    
    
    
\end{document}